\documentclass{ctexart}
\usepackage{amsmath}
\usepackage{amsmath}

\begin{document}
	
	% 矩阵环境,用&分隔列,用\\分隔行 
	\[
		\begin{matrix}
			0&1\\
			1&0
		\end{matrix}\qquad
	% pmatrix 环境
	\begin{pmatrix}
		0&-i\\
		i&0 
	\end{pmatrix} \qquad
	% bmatrix 环境
\begin{bmatrix}
	0&-i\\
	i&0 
\end{bmatrix} \qquad
	% Bmatrix 环境
\begin{Bmatrix}
	0&-i\\
	i&0 
\end{Bmatrix} \qquad
	% Vmatrix 环境
\begin{Vmatrix}
	0&-i\\
	i&0 
\end{Vmatrix} \qquad
\]
% 可以使用上下标 
\[
A=\begin{pmatrix}
	a_{11}^2&a_{12}^2&a_{13}^2\\
	0&a_{22}&a_{23}\\
	0&0&a_{33}
\end{pmatrix}
\]

% 常用省略号:\dots,\vdots,\ddots 
\[
	A=\begin{bmatrix}
		a_{11}&\dots&a_{1n} \\
		&\ddots & \vdots \\ 
		0& & a_{nn}
	\end{bmatrix}_{n \times n}
\]

\[
A=\begin{bmatrix}
	a_{11}&\dots&a_{1n} \\
	\vdots &\ddots & \vdots \\ 
	0& \dots & a_{nn}
\end{bmatrix}_{n \times n}
\]

% 分块矩阵(矩阵嵌套) 
11111111
\[
	\begin{pmatrix}
		\begin{matrix}
			1&0\\0&1
		\end{matrix}
		& \text{\Large 0} \\
		\text{\Large 0} & \begin{matrix}
			1&0\\0&-1 
		\end{matrix}
	\end{pmatrix}
\]
% 三角矩阵 
\[
\begin{pmatrix}
	a_{11}&a_{12}&\cdots &a_{1n} \\
	&a_{22}&\cdots&a_{2n} \\
	& & \ddots & \vdots \\
	\multicolumn{2}{c}{\raisebox{1.3ex}[0pt]{\Huge 0}}
& & a_{nn}
	
\end{pmatrix}\]
% 行内小矩阵(smallmatrix ) 环境 
复数 $z=(x,y)$也可以用矩阵 
\begin{math}
	\left(% 需要手动加上左括号
	\begin{smallmatrix}
	x&-y\\y&x 
	\end{smallmatrix}
\right) % 需要手动加上右括号 
\end{math}

% 跨列的省略号 :\hdotsfor{<列树>} 
\[
	\begin{pmatrix}
		1&\frac 12&\dots & \frac 1n \\ 
		\hdotsfor{4} \\
		m&\frac m2 & \dots & \frac mn 
	\end{pmatrix}
\]

%array 环境 (类似于表格环境tabular ) 
\[
	\begin{array}{r|r} 
		\frac12 & 0 \\ \hline 
		0& -\frac abc \\ 
	\end{array}
\]
% 用array 环境构造复杂矩阵 
 \[
 	\begin{array}{c@{\hspace{-5pt}}1} 
 \]
\end{document}